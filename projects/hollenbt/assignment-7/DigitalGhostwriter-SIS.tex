\documentclass[12pt]{article}
%\usepackage{times}
\usepackage{cite}
\usepackage{graphicx}
\usepackage{url}
\setlength{\parskip}{1em}
\setlength{\parindent}{0em}
%this is a comment
\title{CS 361 \\ Implementation 2 Assignment: \\ Digital Ghostwriter}
\author{Thomas Hollenberg (hollenbt), Zachary Thomas (thomasza),\\ Amar Raad (raadv), Jared Tence (tencej), \\  Zech DeCleene (decleenz)}
\date{March 11th, 2018}

\begin{document}
\maketitle
\newpage
\tableofcontents

\newpage

\section{Product Release}

The latest version of Digital Ghostwriter can be downloaded from this link:\newline
\url{https://drive.google.com/open?id=1x6jxEBkFOGrHUpeS50K7cnCH7CgmZXfc}

We are including a demo video in case you have any issues with setup or installation:\newline
\url{https://media.oregonstate.edu/media/t/0_7w8il9gz}

Once downloaded you should extract the contents of the zip file, all the contents of the program will be contained within the dgw directory.
Within the dgw directory you can find a file called README that gives in depth instructions on how to install/setup and run Digital Ghostwriter. 

Digital ghostwriter is designed to be run on a Linux machine that is using some graphical desktop environment (We used CentOS 7, with Gnome).

\textbf{Installation / Setup:}

1. python3.6 or higher must be installed on your system.

2. Install tensorflow on your system at the directory ~/venvs/tensorflow.
Guide for CentOS 7: https://gist.github.com/thoolihan/28679cd8156744a62f88
Guide for other: https://www.tensorflow.org/install/

3. Extract the contents of the dgw.zip file. You have completed installation / setup.

\textbf{Running Digital Ghostwriter:}

To run Digital Ghostwriter open the dgw directory and run the command: ./exeDGW

Alternately if you installed tensorflow to a directory that was not listed above run the following commands: \newline
source ~/(location of tensorflow)/bin/activate \newline
python3.6 dgw.py

\newpage
\textbf{Viewing genre and authors:}

From the main menu select "genre", here you will see current completed genres (You should see comedy, horror, and science fiction).
Next click on a genre to see the authors for that genre. 

\textbf{Creating a story:}

From the main menu select "write", you will greeted with a page that allows you to select a genre, a word count, and a file name. You must enter a valid value for each or else the program will not allow you to continue. 

Once you are satisfied with your selections you must hit done to continue. This may take up to five minutes based on the computer you are using.

When you see a page that says that the story has been written and saved, you may exit the program (red button in the top right corner of the screen) or continue to write more stories. 

Your finished story can be found in the "stories" folder, it will be named (the name you entered).txt. It will match the word count you entered, and be written based on the genre you selected. 

\section{User Story}

\subsection{Comedy Genre}
\begin{itemize}
\item Team: Jared, Zech, and Amar
\item Problems: Deciding which authors to include and sourcing their novels. Additionally, we had trouble with unusual characters in some of the texts, which were unable to be interpreted correctly. Training the model took too long in the virtual machine, and had to be done on one of our desktops with an external graphics card that could run tensorflow-gpu. The trained model was too large to push to GitHub, so we had to find an alternate way to share the model with the rest of the team.
\item Time Cost: 2 unit 
\item Status: Implemented/Tested
\item Remaining: Nothing. The model successfully produces output in the comedy style.
\item UML sequence diagram: There was no UML sequence diagram for this user story.
\end{itemize}

\subsection{Science Fiction Genre}
\begin{itemize}
\item Team: Jared, Zech, and Amar
\item Problems: Deciding which authors to include and sourcing their novels. Training the model took too long in the virtual machine, and had to be done on one of our desktops with an external graphics card that could run tensorflow-gpu. The trained model was too large to push to GitHub, so we had to find an alternate way to share the model with the rest of the team.
\item Time Cost: 1 unit 
\item Status: Implemented/Tested
\item Remaining: Nothing. The model successfully produces output in the science fiction style.
\item UML sequence diagram: There was no UML sequence diagram for this user story.
\end{itemize}

\subsection{Name Story}
\begin{itemize}
\item Team: Tommy and Zach
\item Problems: We had to ensure that the suggested name was a valid Unix filename, which included checking that its length be less than 256 and consist only of allowed characters, and prevent the story from being written if these criteria were not met.
\item Time Cost:  1 unit
\item Status: Implemented/Tested
\item Remaining: Nothing.
\item UML sequence diagram: The UML sequence diagram was useful and contained everything we needed for the implementation of this user story.
\end{itemize}

\subsection{Select Word Count}
\begin{itemize}
\item Team: Tommy and Zach
\item Problems: We had to ensure that the specified word count was a valid positive number, and prevent the story from being written if this criterion was not met.
\item Time Cost: 1 unit 
\item Status: Implemented/Tested
\item Remaining: Nothing.
\item UML sequence diagram: The  UML sequence diagram was useful and contained everything we needed for the implementation of this user story.
\end{itemize}

\subsection{Write Story}
\begin{itemize}
\item Team: Everyone
\item Problems: Writing the story was easy once we implemented the two previous user stories. Python makes creating child processes incredibly easy. By simply checking the return value, we could easily determine whether to go to the success page or generate an error message and return to the write page.
\item Time Cost: 1 unit 
\item Status: Implemented/Tested
\item Remaining: Nothing.
\item UML sequence diagram: The  UML sequence diagram was useful and contained everything we needed for the implementation of this user story.
\end{itemize}

\section{Design changes and rationale}

The customers decided to consolidate the "writing settings page" with the "file naming page" to keep all settings on one page and make it quicker and more user friendly to create a new story. 

In addition we had noticed that creating a story on some computers could take a long time and clicking the done button after naming the story could make the user feel as if the program had crashed or was frozen. This lack of user feedback felt unintuitive so we added an additional page that states that the program is currently writing the file and to please wait, as soon as the file is done writing the page changes to let the user know that the writing was completed.   
 
We also realized that there is a lot of potential to enter invalid data into the fields for word count and file name so we designed a error pop up that alerts the user of any errors they have made and prevents them from continuing until they have fixed the error(s).

We successfully implemented all of the user stories planned for this week, so no schedule changes were necessary.

As for the estimates of user story effort, we noticed that the genres took about 50 percent more units of effort than we expected, we ended up having trouble finding source material, and some of the source material had special characters that did not transfer well to a Linux system. We ended up resolving all of these issues for the final product. 

\section{Refactoring}

We renamed our Page subclasses to be more descriptive, and removed one that had become obsolete. This also required re-indexing of a static array of pages. Additional refactoring, perhaps moving some shared methods from the subclasses to the Page superclass, would be desirable if there was more time and this project were to continue past the end of this week. 

As a team we realize there is plenty of potential to refactor our code, though  our team lacks experience with python which limits the effectiveness we have when refactoring in a limited time. We have focused on making our code readable and understandable at the cost of some redundant code.

\section{Tests}

\subsection{Verify\_Fields (Unit Test)}

The verify\_fields method verifies that three variables are valid before attempting to write a story: word count, story name, and genre selection. These variables are independent of each other (as long as the others are valid, so the program is not interrupted), so they were tested independently.

\subsubsection{Word Count}
\begin{itemize}
\item Input Values: Selected genre, valid story name, and a string of characters for the word count, not necessarily numeric. The input space is too large for exhaustive testing, so we chose strings from different regions of the input space. For example, invalid strings such negative numbers, floating point numbers, numbers larger than one million, and any string containing a non-numeric character. Only positive integers less than one million should be valid.
\item Test Execution: The verify\_fields method was called by clicking on the "Done" button.
\item Test Results: A valid string was expected to generate no error message and proceed to writing the story. An invalid string was expected to generate an error message describing the requirements for the word count (i.e. positive integer less than a million).
\end{itemize}

\subsubsection{Story Naming}

\begin{itemize}
\item Input Values: Selected genre, valid word count, and story name string. As with the word count, the input space is too large for exhaustive testing, so we chose strings from different regions of the input space. We required that story names be between 1 and 255 characters, inclusive, and only allowed alphanumeric characters, underscores, and dashes.
\item Test Execution: The verify\_fields method was called by clicking on the "Done" button.
\item Test Results: A valid story name was expected to generate no error message and proceed to writing the story. An invalid string was expected to generate an error message describing the requirements for the story name (i.e. length requirements and allowed characters).
\end{itemize}

\subsubsection{Genre Selection}

\begin{itemize}
\item Input Values: Valid word count, valid story name, and a selected genre. Various genres were selected by clicking on the genre buttons on the WriteSpecs page. The empty string was also tested (by clicking "Done" without selecting a genre). Unlike the other two variables, the input space is very small (since there are only a few potential genres, plus no genre), so exhaustive testing is possible.
\item Test Execution: The verify\_fields method was called by clicking on the "Done" button.
\item Test Results: Any selected genre was expected to generate no error message and proceed to writing the story. If no genre was selected, the expected result was to generate an error message informing the user to please select a genre.
\end{itemize}

\subsection{Write Story (Integration Test)}

\begin{itemize}
\item Input Values:  Integer for word count (1000), Test for name, and either
Horror, Comedy, or Science Fiction for Genre. Word Count and Name are kept constant to only see the effects of a change in Genre
\item Test Execution: The test will check if the output files words exist within the English language. It will go through every word within the output file and check whether they exist in an online dictionary. If there exists words within the written story that don't exist in the online dictonary then the test returns FAILED. If the test concludes and all words in the document are in the online dictonary, then the test returns SUCCESS. 
\item Test Results: All test results return SUCCES.
\end{itemize}

\subsection{Word Count / File name (Integration Test)}

\begin{itemize}
\item Input Values: Integer for word count (1(min), 50, 500, 1000000(max)),\newline
Strings for name (A, WORD, MULTIPLEWORDS, EXTRAMULTIPLEWORDS),\newline
Genre set to (Horror) to prevent changes in output based on changes to genre.
\item Test Execution: This test focuses on comparing input with output and confirming the combined results of both word count and file name returns expected output. The testing script runs the program and fills out all possible combinations of the input values (see above). It then runs a word count command on the given file name. If that file name is not found then that test is returned as FAILED, if that file name is found and the word count returned is not equal to the word count entered as input then that test returns as FAILED. If the file name is found and the word count returned is equal to the word count entered as input then that test returns SUCCESS.
\item Test Results: All test values returned SUCCESS.
\end{itemize}

\subsection{Genre / Write Story (Integration Test)}

\begin{itemize}
\item Input Values: Genre (Horror, Comedy, Science Fiction), \newline
Word Count (1000), we want a high word count to allow us to make accurate comparisons to the source text. We do not change the word count.
Name (TextTest), name doesn't change to prevent changes in name changing our output.  
\item Test Execution: This test focuses on seeing if words in our output text match words in the source used to create the genre model. The testing script runs the program and fills out all possible combinations of the input values (see above). Each instance of the test opens the output text file (TextTest.txt) and compares each word in the document with all files in the genre folder that is its source text. If it fails to find less than 90 percent of all of its words in the source text then the test is returned as FAILED, if the test is able to find 90 percent or greater matches it returns SUCCESS.
\item Test Results: All test values returned SUCCESS.
\end{itemize}

\section{Meeting Report }

\subsection{Next week's schedule}
Monday and early Tuesday the team is working on finalizing and practicing our presentation of Digital Ghostwriter.
Late Tuesday or Thursday we will be presenting Digital Ghostwriter to the class.

At this point there are no more assignments for working on Digital Ghostwriter, but if we were to continue this would be our schedule for development this upcoming week:

User created genre\newline
Due: Next week.\newline
Tasks:\newline
1. Create interface for new genre creation.\newline
2. Dynamically generate new model with user selected name.\newline
3. Dynamically display genre in genre and write menus.\newline
4. Enable usage of model to create stories.\newline
Prerequisites:\newline
Genre must be viewable before this feature can be implemented.\newline
Effort estimate: 8 units.

Delete genre\newline
Due: Next week.\newline
Tasks:\newline
1. Remove genre folder from file system, as well as the contained model and
training material.\newline
2. Remove genre from GUI display, and destroy genre object in memory.\newline
Prerequisites:\newline
Genre must be viewable before this feature can be implemented.\newline
Genre to be removed is in the genre list.\newline
Effort estimate: 2 units.

Delete authors\newline
Due: Next week.\newline
Tasks:\newline
1. Remove author from author list, and retrain genre.\newline
2. Remove author from GUI.\newline
Prerequisites:\newline
Genre must be viewable before this feature can be implemented.\newline
Authors must be viewable before this feature can be implemented.\newline
Author to be removed is in the author list.\newline
Effort estimate: 4 units.

Add authors\newline
Due: Next week.\newline
Tasks:\newline
1. Design interface to allow users to add authors to genre.\newline
2. Update genre model based on new authors input.\newline
Prerequisites:\newline
Genre must be viewable before this feature can be implemented.\newline
Authors must be viewable before this feature can be implemented.\newline
Effort estimate: 4 units.

\subsection{This week's progress:}
Added more functionality to our Digital Ghostwriter program, including the five planned user stories: comedy genre, science fiction genre,  naming story, selectable word count, generate story as text document. 

\subsection{Goals for next week:}
Tuesday meeting at 3pm at JOHN to prepare for our presentation. Have all members present for presentation. 

\subsection{Contribution of team members:}
All team members gave feedback and worked on our assignment.

\subsection{Were customers able to meet with the team:}
Yes. Team including customers met to complete assignment, as well as continued to refine goals, schedule, and work on presentation. 

\end{document}
